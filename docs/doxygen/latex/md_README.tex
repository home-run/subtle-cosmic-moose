\href{http://waffle.io/home-run/subtle-cosmic-moose}{\tt }

\subsection*{$\sim$/run (H\+O\+ME R\+UN)}

\subsection*{\#\#\+An application for planning a baseball fan\textquotesingle{}s dream vacation. }

\subsubsection*{Git flow and branching scheme}


\begin{DoxyEnumerate}
\item {\bfseries Always} branch from {\ttfamily develop}.
\item Branch naming scheme is defined as such\+:
\begin{DoxyItemize}
\item {\ttfamily $<$issue number$>$-\/brief-\/description}
\item For example, \href{https://waffle.io/home-run/subtle-cosmic-moose/cards/56fa022814437c0e00ba6c1e}{\tt issue \#1 is to create a R\+E\+A\+D\+ME file}, so my branch name is {\ttfamily 1-\/readme}.
\end{DoxyItemize}
\item Include a \textquotesingle{}\#\textquotesingle{} followed by the issue number in every commit.
\begin{DoxyItemize}
\item For example, commits to this branch will look like this\+: {\ttfamily \mbox{[}\#1\mbox{]} Added stuff to readme.}
\end{DoxyItemize}
\item Never work on a branch without an associated issue. If you don\textquotesingle{}t see the issue on waffle, create one.
\item Always push a branch immediately after creating it. This is to correctly update issues on waffle, and also to let other teammates know what you\textquotesingle{}re working on.
\item Pull-\/requests must be reviewed by all members of the team before they are approved for merge.
\item Delete the feature branch once a pull-\/request is merged.
\end{DoxyEnumerate}

\subsubsection*{A mini command-\/line git tutorial (for our purposes for this project)}

0. Clone the repo.
\begin{DoxyItemize}
\item {\ttfamily git clone \href{https://github.com/home-run/subtle-cosmic-moose.git}{\tt https\+://github.\+com/home-\/run/subtle-\/cosmic-\/moose.\+git}}
\item {\ttfamily cd subtle-\/cosmic-\/moose}
\end{DoxyItemize}

Switch to branch {\ttfamily develop}.
\begin{DoxyItemize}
\item {\ttfamily git checkout develop}
\end{DoxyItemize}

Make sure to do a {\ttfamily git pull} before branching, so you always have up-\/to-\/date code.
\begin{DoxyEnumerate}
\item Create/switch to your branch. (This one command does both.)
\begin{DoxyItemize}
\item {\ttfamily git checkout -\/b 66-\/a-\/brief-\/description}
\end{DoxyItemize}
\item Push the newly created branch.
\begin{DoxyItemize}
\item {\ttfamily git push -\/u origin 66-\/a-\/brief-\/description}
\end{DoxyItemize}
\item Start coding!
\item Add your changes to the staging area to prepare for a commit.
\begin{DoxyItemize}
\item {\ttfamily git add my\+\_\+new\+\_\+file.\+cpp} for files that don\textquotesingle{}t exist on the repo yet.
\item {\ttfamily git add -\/u} for files that already exist on the repo that have been updated.
\item You can use the command {\ttfamily git status} to view the staging area at any time.
\end{DoxyItemize}
\item Make a commit. Make sure to include \textquotesingle{}\#\textquotesingle{} followed by your issue number somewhere in the commit message.
\begin{DoxyItemize}
\item {\ttfamily git commit -\/m \char`\"{}\mbox{[}\#66\mbox{]} Mashed keys until it compiled. Don\textquotesingle{}t bother testing it. It\textquotesingle{}s fine, I swear.\char`\"{}}
\end{DoxyItemize}
\item Push your changes.
\begin{DoxyItemize}
\item If you used the command in {\bfseries step 4}, this is simply {\ttfamily git push} (The {\ttfamily -\/u} flag sets the upstream so you don\textquotesingle{}t have to explicitly set it for every push).
\item Otherwise, the command is {\ttfamily git push origin 66-\/a-\/brief-\/description}
\end{DoxyItemize}
\item Drink more coffee. \+:coffee\+:
\end{DoxyEnumerate}

\subsubsection*{A Brief Doxygen Style Guide}

{\bfseries note\+:} {\itshape A\+LL code must be doxygen commented.} {\bfseries If your code isn\textquotesingle{}t commented, it will not be approved for merge.}

{\bfseries In QT, typing} {\ttfamily /$\ast$$\ast$ + return} {\bfseries above a declaration will produce a doxygen comment block.}

Full commenting manual can be found \href{http://www.stack.nl/~dimitri/doxygen/manual/docblocks.html}{\tt here}.

\paragraph*{Things to note about Doxygen comments}


\begin{DoxyItemize}
\item Comments can come before the code item.
\item For class members and parameters they may also come after them.
\item They may be either brief (one line) or detailed or both.
\item Put the reference documentation type comments (class and method descriptions) {\bfseries in the .h file} and {\itshape not} in (or, at least, in addition to) the .cpp files.
\end{DoxyItemize}

\paragraph*{Brief comment before}

Add an extra \char`\"{}/\char`\"{}


\begin{DoxyCode}
1 /// This method does something
2 void DoSomething();
\end{DoxyCode}


\paragraph*{Detailed comment before}

Add an extra \char`\"{}$\ast$\char`\"{}


\begin{DoxyCode}
1 /** This is a method that does so
2   * much that I must write an epic 
3   * novel just to describe how much
4   * it truly does. */
5 void DoNothing();
\end{DoxyCode}



\begin{DoxyItemize}
\item the intermediate leading \char`\"{}$\ast$\char`\"{}s are optional.
\end{DoxyItemize}

\paragraph*{Brief comment after}

Add an extra \char`\"{}/$<$\char`\"{}


\begin{DoxyCode}
1 void DoSomething(); ///< This method does something
\end{DoxyCode}


\subsubsection*{Detailed comment after}

Add an extra \char`\"{}$\ast$$<$\char`\"{}


\begin{DoxyCode}
1 void DoNothing(); /**< This is a method that does so
2   * much that I must write an epic 
3   * novel just to describe how much
4   * it truly does. */
\end{DoxyCode}



\begin{DoxyItemize}
\item the intermediate leading \char`\"{}$\ast$\char`\"{}s are optional.
\end{DoxyItemize}

\paragraph*{Example .h file}

Below is a full example.

``` /$\ast$$\ast$
\begin{DoxyItemize}
\item 
\end{DoxyItemize}